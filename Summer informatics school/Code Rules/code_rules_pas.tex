\documentclass[11pt,a4paper,oneside]{article}

\usepackage[english,russian]{babel}
\usepackage[T2A]{fontenc}
\usepackage[utf8]{inputenc}
\usepackage[russian]{olymp}
\usepackage{graphicx}
\usepackage{expdlist}
\usepackage{mfpic}
\usepackage{amsmath}
\usepackage{amssymb}
\usepackage{comment}
\usepackage{listings}
\usepackage{epigraph}
\usepackage{url}
\usepackage{ulem}

\DeclareMathOperator{\nott}{not}

\begin{document}

\renewcommand{\t}[1]{\mbox{\texttt{#1}}}
\newcommand{\s}[1]{\mbox{``\t{#1}''}}
\newcommand{\eps}{\varepsilon}
\renewcommand{\phi}{\varphi}
\newcommand{\plainhat}{{\char 94}}

\newcommand{\Z}{\mathbb{Z}}
\newcommand{\w}[1]{``\t{#1}''}


\binoppenalty=10000
\relpenalty=10000

\contest{Правила оформления кода}{ЛКШ 2015 Июль}{B'}

\begin{LARGE} \textbf{Требования к оформлению кода на Pascal} \end{LARGE}
\newline
\subsection*{Форматирование}
\begin{enumerate}
    \item Используйте два или четыре пробела для отступов. 
          Недопустимо использовать символ табуляции для отступов. 
          Отступы внутри одной программы должны быть одинакового размера.
    \item Отступы должны присутствовать внутри любого блока.
          После заголовка блока, нужно перевести строку.
    \item Объявление переменных должно быть максимально близко к месту их использования.
    \item Все блоки описаний \texttt{(var, const и т. д.)} использовать не более одного раза.
    \item Если вы используете константу более одного раза, объявите ее, как таковую.
    \item Функции разделяйте одной пустой строкой. 
          Смысловые блоки внутри одной функции разделяйте пустрой строкой.
    \item Cкобки не отделяются пробелами с внутренней стороны. 
          Между функцией и ее аргументами пробел не ставится.
          \\ \texttt{ spam(ham[1]); \hbox{ } \hbox{ } \hbox{ } // Правильно}
          \\ \texttt{ spam( ham [ 1 ] ); // Неправильно}
    \item Перед запятой пробел не ставится, после --- ставится.
          \begin{verbatim}
writeln(x, y);  // Правильно
            
writeln(x , y); // Неправильно
writeln(x,y);   // Неправильно  
          \end{verbatim}
    \item Всегда окружайте следующие бинарные операторы ровно одним символом пробела с каждой стороны:
          \begin{enumerate}
            \item присваивания \texttt{(:=)}
            \item сравнения \texttt{(=, <, >, <>, <=, >=)}
            \item логические \texttt{(and, or, not)}
            \item арифметические \texttt{(+, -, *, /, div, mod)}
          \end{enumerate}              
    \item Не располагайте несколько инструкций в одной строке. Разнесите их по разным строкам.
          \\ \texttt{x = 35;} 
          \\ \texttt{func(10); \hbox{ } \hbox{ } \hbox{ } \hbox{ } // Правильно}
          \\ \texttt{x = 35; func(10); // Неправильно}
    \item Не располагайте блок из нескольких инструкций на одной строке сразу 
          после \texttt{if}, \texttt{while} и т. д.
    \item Ширина строки не должна превышать 80-100 символов.
                
\subsection*{Комментарии}
    
     \item Комментарии, противоречащие коду, хуже, чем их отсутствие.
     \item Располагайте однострочные комментарии после кода в той же строке и отделяйте
           их от кода не менее чем двумя пробелами. Однострочные комментарии должны начинаться с <<//>> и
           одного пробела, а многострочные с <<{>>  и одного пробела. Также многострочный комментарий нужно
           закрыть <<}>>. 

\subsection*{Имена}

     \item Не используйте символы <<l>>, <<O>>, и <<I>> как имена переменных. 
           В некоторых шрифтах они могут быть очень похожи на цифры.
     \item Название переменной должно содержать информацию о ее назначении, 
            не допускается использование названий из одной буквы, за исключением счетчиков в циклах
            \texttt{(i, j, k)} и переменных, хранящих размер входных данных \texttt{(n, m)}.
     \item Давайте переменным говорящие английские имена, не используйте транслит.
           \\ \texttt{read(num\_letters); // Правильно}
           \\ \texttt{read(kolvo\_bukv); \quad // Неправильно}
     \item Имена переменных и функций должны содержать только маленькие буквы. Слова
           разделяются символами подчёркивания. Примеры:
           \texttt{name}, \texttt{name\_with\_several\_words\_in\_it}
           Допускается также записывать начальные буквы каждого слова с заглавной буквы,
           при этом символы подчёркивания не ставятся: \texttt{wordLength}  
     \item Имена констант должны содержать только заглавные буквы. Слова разделяются
           символами подчёркивания. Примеры: \texttt{NAME}, \texttt{NAME\_WITH\_SEVERAL\_WORDS\_IN\_IT}


\end{enumerate}


\end{document}
